% *****************************************************************************
%
% Author: Xavier Chassin
% Project: Deadpool - Battleship
%
% Documents the algorithms used to play the battleship game
%
% *****************************************************************************

\documentclass[a4paper]{article}
% *****************************************************************************
%
% Packages and their options
%
% *****************************************************************************
\usepackage[utf8]{inputenc}
\usepackage[a4paper,margin=2.54cm]{geometry}  % Geometry of the pages
\usepackage[nonumberlist]{glossaries}

\makeglossary

\begin{document}

\title{Deadpool - Battleship}
\author{Xavier Chassin}
\maketitle

%Term definitions
\newglossaryentry{board_size}{name=$s$, description={Side length of the board}}

%Print the glossary
\printglossary[title=Notations]

% ------------------------------
% Document start
% ------------------------------

\section{Introduction} % (fold)
\label{sec:introduction}

This document presents the \emph{Deadpool Battleship} project. The focus is on the algorithms used to play a \emph{Battleship} game, as well as their implementation. Those will hopefully be implemented in different languages:
\begin{itemize}
\item Go
\item Python
\item And potentially others (C++, Haskell...)
\end{itemize}

% section introduction (end)

\section{Playing randomly} % (fold)
\label{sec:playing_randomly}

To setup game and have a working implementation of the server, I needed a first player. The easiest, non completely trivial algorithm is the random one. It randomly picks a tile amongst all the tiles on which it has not fired yet.

% section playing_randomly (end)

\section{Scoring the tiles} % (fold)
\label{sec:scoring_the_tiles}

The random player is helpful to get the server started, however it is not exactly funny from a algorithm point of view. An idea of a battleship algorithm would be to score each tile and then fire on the tile with the highest score.

\subsection{Distance based} % (fold)
\label{sub:distance_based}

To score a tile, we can use the distance from that tile the closest known (i.e. fired upon) tile. When playing a battleship game, if you fire on a tile that happens to only be the sea, then the adjacent tiles are less likely to be ship tiles.

\subsubsection{Ship adjacent tiles} % (fold)
\label{ssub:tiles_adjacent_to_a_ship}

The obvious drawback of such a scoring function is that if we actually fire on a ship, the adjacent tiles will have a minimal score and hence will not be fired upon. To overcome this drawback, we set the score $s = \max_{t \in \mathcal{T}}s_t$ to the tiles adjacent to a ship.

% subsubsection tiles_adjacent_to_a_ship (end)

\subsubsection{A bit of randomness} % (fold)
\label{ssub:a_bit_of_randomness}

In order to avoid having tiles with the same score, we add a random ``error'' to the distance:
\begin{equation*}
s_t = \min_{t \in \mathcal(F)}d(t, t') + \epsilon
\end{equation*}
where $\epsilon \sim \mathcal{N}(0, 1)$. This also has the advantage of enhancing exploration of the board.

% subsubsection a_bit_of_random (end)

% subsection distance_based (end)

% section scoring_the_tiles (end)

\end{document}
